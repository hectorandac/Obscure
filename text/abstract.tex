\addcontentsline{toc}{section}{Abstract}	% Add Abstract section to toc
\section*{Abstract} 						% Section name

In the evolving landscape of object detection, the advent of the YOLO (You Only Look Once) methodology has marked a significant leap forward for real-time applications, streamlining the process with its single-pass detection capabilities. Despite its advancements, the dynamic and often unpredictable nature of real-world environments, especially when operating on edge devices like security cameras, presents a pressing challenge for optimization. Addressing this, our research introduces the "Gated Scene-Specific YOLO," a novel adaptation of the YOLO framework, incorporating a dynamic gating mechanism aimed at enhancing computational efficiency with minimal compromise on the model's detection prowess.

Traditional YOLO architectures, while robust, are predisposed to processing vast amounts of data, much of which may be extraneous or irrelevant to the task at hand. This not only leads to unnecessary computational overhead but also hinders the deployment of such models in environments where resources are limited. Our Gated Scene-Specific YOLO methodology seeks to alleviate this issue by integrating a mechanism that dynamically adjusts the activation of neural pathways based on the relevance to the observed scene. Through a meticulous process of gate generation and analysis during the training phase, our approach identifies and deactivates neural pathways that are consistently inactive across specific environmental conditions. This strategic deactivation allows the model to shed redundant computational weight, thus becoming more streamlined and efficient for the task it is deployed to perform.

The core of our research lies in demonstrating the practicality of dynamically tuning deep learning models to their operational context, significantly reducing the computational load while minimally impacting the detection accuracy. Our empirical results showcase that the Gated Scene-Specific YOLO not only elevates processing speeds but also upholds a high standard of accuracy, making it a compelling solution for real-time object detection across a diverse array of settings. This contribution is particularly relevant for the deployment of object detection models in resource-constrained devices, where optimizing the balance between efficiency and performance is paramount.

In summary, the Gated Scene-Specific YOLO represents a meaningful stride towards more adaptable and resource-efficient object detection solutions. By tailoring model processing pathways to the specific demands of the environment, this research paves the way for the development of highly optimized, context-aware deep learning models, thereby enhancing the applicability and effectiveness of real-time object detection systems in fixed and scene specific settings.

\clearpage
