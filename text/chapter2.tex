\section{Related Work}  

The exploration of efficiency within neural network architectures, particularly for object detection in computationally constrained environments, is a significant area of research. This section reviews various methodologies and developments that have shaped the current landscape of efficient neural network design, highlighting the relevance and novelty of our approach.

\subsection{Sparsity and Conditional Computation}

A fundamental concept in neural network efficiency is the integration of sparsity and conditional computation. Sparsity involves activating only a subset of neural pathways for a given input, reducing computational overhead without sacrificing the model's ability to represent complex functions. This principle is akin to decision trees, where decisions lead to a subset of possible states, thus not exploring all branches for a given input.

The concept of conditional computation extends this idea further by introducing mechanisms that allow a neural network to adapt its computation pathways dynamically based on the input. This adaptability ensures that only the most relevant parts of the network are engaged during the forward pass, which not only saves computational resources but also helps in reducing over-fitting by limiting the effective capacity of the model based on the complexity of the input.

Conditional computation extends this idea by dynamically engaging only the most relevant parts of the network during the forward pass, saving computational resources and reducing over-fitting. Introduced by Bengio et al. \cite{bengio2013deep}, this mechanism allows for deeper and more complex models by judiciously allocating computational resources.

Several approaches have been proposed to implement sparsity and conditional computation. For example, Chen et al. introduced GaterNet \cite{chen2019you}, where a gating network generates binary gates to selectively activate filters in the backbone network based on input. Their experiments on CIFAR and ImageNet datasets show significant improvements in model performance and efficiency.

Veit and Belongie \cite{veit2020convolutional} propose adaptive inference graphs, dynamically selecting parts of the network to use for each input, reducing computational requirements. Liu et al.'s Dynamic Sparse Training (DST) \cite{liu2020dynamic} represents a leap from static sparsity models, dynamically adjusting network architecture during training to optimize model capacity and resource allocation.

Kalwar et al. \cite{kalwar2023gdip} explore object detection in adverse weather conditions using gated image processing, highlighting the adaptability and efficiency of gated architectures, presenting potential comparative studies for our gated approach.

\subsection{Neural Network Pruning}

Neural network pruning reduces model complexity by eliminating redundant weights, enhancing computational efficiency without significantly sacrificing accuracy. This technique alleviates storage and computational burdens, leading to faster inference times and reduced energy consumption, crucial for deployment on resource-constrained devices.

\textbf{Gradient-based Pruning.} Zhang et al. \cite{zhang2023neural} leverage gradient descent to systematically identify and eliminate less critical connections, preserving model performance while streamlining its architecture.

\textbf{Iterative Pruning and Retraining.} Han et al. \cite{han2016deep} propose an iterative approach, periodically removing weights deemed least important, followed by retraining to maintain performance. This method achieves substantial model compression while maintaining high accuracy with fewer parameters.

Both gradient-based and iterative pruning methods enhance model efficiency but require careful calibration and additional development steps, such as determining optimal pruning thresholds and managing retraining processes.

\subsection{Quantization Techniques for Neural Network Efficiency}

Quantization reduces numerical parameter precision from floating-point to lower-bit representations, significantly reducing model size and computational complexity while maintaining acceptable accuracy. Krishnamoorthi \cite{krishnamoorthi2018quantizing} provides an overview of quantization techniques, including post-training quantization and quantization-aware training.

\textbf{Efficient Integer-Arithmetic-Only Inference.} Jacob et al. \cite{jacob2018quantization} introduce a framework for efficient inference in integer-arithmetic-only environments, enabling the use of optimized hardware accelerators.

\textbf{Hardware-Aware Automated Quantization.} Wang et al. \cite{wang2019haq} propose HAQ, a method using reinforcement learning to determine optimal mixed precision quantization policies based on target hardware characteristics.

While quantization methods offer significant improvements in model efficiency, they often require careful tuning of quantization parameters to avoid substantial accuracy loss. Additionally, the static nature of most quantization techniques means that once a model is quantized, its ability to dynamically adjust to varying computational resources or input complexities is limited.

\subsection{Knowledge Distillation for Model Efficiency}

Knowledge distillation transfers knowledge from a larger, complex model (teacher) to a smaller, efficient model (student). Hinton et al. \cite{hinton2015distilling} demonstrate that a smaller model can mimic a larger model's output, enabling high performance on resource-limited devices. Zhang et al. \cite{zhang2019your} explore self-distillation, where a single network serves as both teacher and student, simplifying the distillation process and improving performance.

Knowledge distillation, while effective, often relies on pre-trained large-scale models, which can be a limitation in scenarios where such models are unavailable or computationally prohibitive.

\subsection{Our Contribution}

The pursuit of efficiency in neural network models has prompted innovations such as quantization, knowledge distillation, sparsity, and neural network pruning. These methods aim to optimize performance within computational constraints by reducing parameter precision, transferring knowledge from larger to smaller models, and eliminating less critical connections. However, these techniques often encounter limitations, such as the need for a large teacher model in knowledge distillation or the loss of adaptability in static pruning and quantization methods.

Our approach integrates gating mechanisms with model pruning, specifically designed for the YOLO architecture. Unlike conventional strategies focusing on static model compression, our method uses Improved SemHash to create static gating configurations after training. This innovation substantially decreases computational demands during inference while maintaining high accuracy.

Building on neural network efficiency advancements, we introduce the ``Gated YOLO.'' This method incorporates gating and model pruning principles, customized for the YOLO architecture. Our technique for establishing static gating configurations post-training is unique in the literature. This strategy significantly reduces computational requirements during inference, distinguishing our work in the real-time object detection landscape. Through comprehensive comparisons with state-of-the-art models, we highlight the practical benefits and superiority of our approach, especially in resource-constrained scenarios.


\clearpage
