\addcontentsline{toc}{section}{Abstract in Korean}	% Add Korean abstract
\section*{Abstract in Korean} 						% Section name

객체 탐지 분야에서 YOLO(You Only Look Once) 방법론은 단일 패스 탐지 기능으로 실시간 응용 프로그램에 혁명을 일으켰습니다. 그러나 기존의 YOLO 구조는 방대한 양의 데이터를 처리하여 불필요한 계산을 초래하고 자원이 제한된 환경에서의 배포를 제한합니다. 우리 연구는 YOLO 프레임워크를 개조한 ``Gated YOLO''를 도입하여 계산 효율성을 높이면서 탐지 정확도를 유지하는 게이트 메커니즘을 통합합니다.

이 방법은 관찰된 장면의 관련성에 따라 신경 경로의 활성화를 조정하는 메커니즘을 통합합니다. 훈련 단계에서 우리의 접근 방식은 특정 환경 조건에서 일관되게 비활성된 신경 경로를 식별하고 비활성화합니다. 이러한 전략적 비활성화는 불필요한 계산 부담을 줄여 지정된 모델 작업에 더 효율적으로 만듭니다.

실험 결과, 우리의 접근 방식은 탐지 정확도에 미치는 영향을 최소화하면서 계산 부담을 크게 줄여 처리 속도를 높이는 것으로 나타났습니다. 이는 보안 카메라 시스템과 같은 고정된 자원이 제한된 환경에서 실시간 객체 탐지에 효과적인 해결책이 됩니다.

요약하자면, Gated YOLO는 더 효율적인 객체 탐지 솔루션을 향한 중요한 진전을 나타냅니다. 모델 처리 경로를 특정 환경 요구에 맞춤으로써 이 연구는 정적 설정에서 실시간 객체 탐지 시스템의 적용 가능성과 효과를 향상시킵니다.

\clearpage