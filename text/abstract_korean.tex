\addcontentsline{toc}{section}{Abstract in Korean}	% Add Korean abstract
\section*{Abstract in Korean} 						% Section name

객체 탐지의 진화하는 풍경에서 YOLO(You Only Look Once) 방법론의 등장은 실시간 애플리케이션을 위한 중요한 도약을 표시했으며, 단일 패스 탐지 기능으로 프로세스를 간소화했습니다. 그러나 YOLO의 발전에도 불구하고, 특히 보안 카메라와 같은 엣지 장치에서 운영될 때 실제 세계 환경의 동적이고 종종 예측할 수 없는 성격은 최적화를 위한 절박한 도전을 제시합니다. 이에 대응하여, 우리의 연구는 동적 게이팅 메커니즘을 통합하여 계산 효율성을 최소한의 모델 탐지 능력 저하로 향상시키는 YOLO 프레임워크의 새로운 적용인 "Gated Scene-Specific YOLO"을 소개합니다.

전통적인 YOLO 아키텍처는 강력하지만, 손에 들고 있는 작업과 관련 없거나 불필요한 방대한 양의 데이터를 처리하는 경향이 있습니다. 이는 불필요한 계산 오버헤드로 이어질 뿐만 아니라 리소스가 제한된 환경에서 이러한 모델의 배포를 방해합니다. 우리의 Gated Scene-Specific YOLO 방법론은 관찰된 장면에 대한 관련성에 기반하여 신경 경로의 활성화를 동적으로 조정하는 메커니즘을 통합함으로써 이 문제를 완화하고자 합니다. 훈련 단계에서 게이트 생성 및 분석의 세심한 과정을 통해, 우리의 접근 방식은 특정 환경 조건에서 지속적으로 비활성화되는 신경 경로를 식별하고 비활성화합니다. 이 전략적 비활성화는 모델이 중복된 계산 무게를 벗어던짐으로써 보다 간소화되고 효율적으로 작업을 수행하도록 합니다.

우리 연구의 핵심은 실제 운영 환경에 딥러닝 모델을 동적으로 조정하는 실용성을 보여주며, 탐지 정확도에 최소한의 영향을 미치면서 계산 부하를 상당히 줄입니다. 우리의 실증 결과는 Gated Scene-Specific YOLO가 처리 속도를 높이는 것뿐만 아니라 높은 정확도 표준을 유지함으로써 다양한 설정에서 실시간 객체 탐지를 위한 매력적인 해결책임을 보여줍니다. 이 기여는 특히 효율성과 성능 사이의 균형을 최적화하는 것이 중요한 리소스 제약 장치에서 객체 탐지 모델의 배포에 특히 관련이 있습니다.

요약하자면, Gated Scene-Specific YOLO는 보다 적응 가능하고 리소스 효율적인 객체 탐지 솔루션을 향한 의미 있는 발걸음을 나타냅니다. 환경의 구체적 요구에 모델 처리 경로를 맞춤 설정함으로써, 이 연구는 고도로 최적화된, 상황 인식 딥러닝 모델의 개발을 위한 길을 열며, 고정 및 장면별 설정에서 실시간 객체 탐지 시스템의 적용 가능성과 효과를 향상시킵니다.

\clearpage