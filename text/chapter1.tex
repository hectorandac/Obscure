\section{Motivation and Objectives}  

\subsection{Introduction}

Object detection is a foundational pillar within computer vision, influencing applications ranging from advanced surveillance systems to autonomous driving technologies. The rapid evolution of deep learning methodologies has significantly propelled the field forward, enhancing the accuracy, efficiency, and adaptability of object detection in real-time processing environments. Among these innovations, the YOLO (You Only Look Once) architecture, introduced by Redmon et al. \cite{redmon2016you}, has enabled efficient processing without iterative detection stages. This architecture underscores the potential of modern object detection systems, from high-powered servers to constrained devices such as smartphones and embedded systems like Nvidia's Jetson Platform.

Despite the advancements brought by YOLO and its iterations, challenges remain, particularly in edge computing where computational resources are limited and real-time processing is essential. Recognizing the potential for further optimization, our study explores YOLOv6 \cite{li2022yolov6,li2023yolov6}, a variant designed with a focus on hardware efficiency and real-time applications. YOLOv6's architecture \cite{li2022yolov6,li2023yolov6} serves as an ideal foundation for our investigation due to its performance in hardware-constrained environments.

Our research introduces ``Gated YOLO,'' a novel framework that integrates dynamic gating with model pruning, specifically tailored to the YOLO architecture. Our approach maintains high modularity, facilitating portability to other YOLO architectures.

A cornerstone of our methodology is the adaptation of Improved SemHash \cite{kaiser2018discrete}, initially proposed by Kaiser and Bengio and further refined by Chen et al. \cite{chen2019you}. This technique generates binary gates during model training, enabling selective activation or deactivation of network filters. Through dynamic gate generation and analysis tailored to specific scenes, our method identifies filters that remain consistently inactive. These filters are statically pruned for deployment, allowing the model to operate efficiently by focusing computational resources on active, scene-relevant pathways. The Gater Network \cite{chen2019you}, integral during training for gate determination, is rendered unnecessary during deployment, replaced by pre-determined, statically applied gates that ensure efficiency and specificity in detection.

Our contributions to the YOLO architecture, through the integration of a gating network and Improved SemHash \cite{kaiser2018discrete}, enable precise control over network activity and significant improvements in detection accuracy. Rigorous experimental validation, focusing on metrics such as inference time in milliseconds (ms), recall, precision and mean Average Precision (mAP@0.5:0.95), reveals a notable decrease in inference time for the Gated YOLO model compared to similar sized counterparts, with stable mAP scores.

\subsection{Motivation}

The relentless pursuit of advancements in computer vision, particularly in object detection, is driven by the escalating demands of modern applications. Deep learning architectures like YOLO have significantly narrowed the gap between theoretical possibility and practical implementation, offering a glimpse into the potential of real-time object detection systems. However, as these technologies are increasingly deployed on the edge, the limitations of current models under resource-constrained conditions become apparent. Our work aims to transcend these limitations and push the boundaries of existing object detection frameworks.

Our focus on YOLOv6 \cite{li2022yolov6,li2023yolov6}, known for its balance of speed and accuracy, stems from the critical need for optimization in edge computing scenarios where resources are scarce yet the demand for high-performance computing is high. The drive to refine and enhance the efficiency of such models without compromising their detection capabilities underlines our research. We are particularly inspired by the potential impact of our work on applications requiring real-time analysis and feedback.

\subsection{Objectives}

The primary objective of our research is to develop an optimized version of the YOLOv6 \cite{li2022yolov6,li2023yolov6} architecture, termed ``Gated YOLO,'' which incorporates a gating mechanism to enhance computational efficiency in object detection tasks, particularly in edge computing environments. To achieve this, we aim to:

\paragraph{Implement Gating Mechanism:} Integrate a gating mechanism that selectively activates relevant neural pathways, improving model efficiency.

\paragraph{Optimize Through Model Pruning:} Apply model pruning techniques to eliminate redundant parameters and streamline the model, focusing computational resources on critical tasks.

\paragraph{Leverage Improved SemHash \cite{kaiser2018discrete}:} Utilize the Improved SemHash technique for effective gate generation during training, allowing for precise control over network activity.

\paragraph{Demonstrate Practical Efficacy:} Validate the effectiveness of the Gated YOLO model through extensive testing, focusing on key performance metrics such as inference time in milliseconds (ms), recall, precision and mean Average Precision (mAP@0.5:0.95).

Through these objectives, our research seeks to address the challenges of deploying sophisticated object detection models in edge computing scenarios, offering a pathway to more efficient, accurate, and accessible real-time object detection technologies.

\clearpage

