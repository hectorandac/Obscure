\section{Motivation and Objectives}  

\subsection{Introduction}

Object detection stands as a foundational pillar within the field of computer vision, influencing an extensive range of applications from advanced surveillance systems to the dynamic realm of autonomous driving technologies. The rapid evolution of deep learning methodologies has significantly propelled the field forward, introducing architectures capable of not only enhancing the accuracy of object detection but also its efficiency and adaptability in real-time processing environments. Among these innovations, the YOLO (You Only Look Once) architecture, introduced by Redmon et al. \cite{redmon2016you}, has emerged as a significant contribution, revolutionizing the way real-time object detection is approached by enabling swift and efficient processing without the need for iterative detection stages. This architecture underscores the potential of modern object detection systems, demonstrating remarkable versatility across a variety of computing environments, from high-powered servers to more constrained devices such as smartphones and embedded systems like the ones introduced by the Nvidia Jetson Platform.

Despite the advancements brought forth by YOLO and its subsequent iterations, challenges remain, particularly in the context of edge computing where computational resources are limited, and the demand for real-time processing is paramount. Recognizing the potential for further optimization, our study delves into the exploration of YOLOv6 \cite{li2022yolov6,li2023yolov6}, a variant designed with a keen focus on hardware efficiency and optimized for real-time applications. The architecture of YOLOv6 \cite{li2022yolov6,li2023yolov6} serves as an ideal foundation for our investigation, given its emphasis on performance in hardware-constrained environments. Within such contexts, minor enhancements can lead to substantial gains in processing speed and overall system efficiency, thereby improving the applicability of YOLO-based models in edge devices.

To advance the capabilities of YOLOv6 \cite{li2022yolov6,li2023yolov6}, our research introduces the "Gated YOLO," a novel framework that marries the concept of dynamic gating with the principle of model pruning specifically tailored to the YOLO architecture. That being said our approaches are not limitted to the integration with YOLOv6, given that we maintain a high degree of modularity of its components, we facilitate the ability to port it to other YOLO architectures. 

Model pruning, a technique aimed at reducing the computational burden of neural networks, achieves this by eliminating superfluous or insignificant parameters, thereby refining the network's structure with minimal detriment to its performance. Our approach innovates beyond traditional model pruning by implementing a dynamic gating mechanism that adjusts in real-time during training to the distinctive characteristics of the input scene, thereby optimizing the efficiency of the object detection process with minimal sacrifice in accuracy.

A cornerstone of our methodology is the adaptation of Improved SemHash \cite*{kaiser2018discrete}, a technique initially proposed by Kaiser and Bengio and further refined by Chen et al. \cite*{chen2019you} This approach facilitates the generation of binary gates during the model training phase, enabling selective activation or deactivation of network filters in response to variations in the input. Through dynamic gate generation and subsequent analysis tailored to specific scenes, our method identifies filters that consistently remain inactive. These filters are then statically pruned from the network for deployment, allowing the model to operate more efficiently by focusing computational resources on active, scene-relevant pathways. The Gater Network \cite*{chen2019you}, integral during the training phase for gate determination, is thus rendered unnecessary during actual deployment, replaced by the pre-determined, statically applied gates that ensure both efficiency and specificity in detection.

Our contributions to the YOLO architecture, through the integration of a gating network and the innovative use of Improved SemHash \cite*{kaiser2018discrete}, not only enable precise control over network activity but also herald significant improvements in computational efficiency and detection accuracy. The effectiveness of our approach is substantiated through rigorous experimental validation, focusing on key performance metrics such as floating-point operations per second (FLOPs), frames per second (FPS), and mean Average Precision (mAP@0.5:0.95). Our findings reveal a notable increase in FPS for the Gated YOLO model in comparison to its YOLOv6 counterpart, with a slight compromise on detection robustness, as evidenced by stable mAP scores. This research thus presents a significant step forward in optimizing deep learning models for object detection, especially in scenarios where computational resources are at a premium.

% Add table one \input{tables/table1.tex}	


% Add figure one 
% \begin{figure}[H]			
% 	\centering
% 	\includegraphics[width=\textwidth]{figures/Fig1.png}
%     \caption{Caption for figure one.}
%     \label{fig:figure1}
% \end{figure}

\subsection{Motivation}

The relentless pursuit of advancements in computer vision, specifically within the domain of object detection, has been driven by the escalating demands of modern applications. The inception of deep learning architectures like YOLO has significantly narrowed the gap between theoretical possibility and practical implementation, offering a glimpse into the potential of real-time object detection systems. Yet, as these technologies are increasingly deployed in real-world scenarios, particularly on the edge, the limitations of current models under resource-constrained conditions become apparent. The motivation behind our work is rooted in the desire to transcend these limitations, pushing the boundaries of what is possible with existing object detection frameworks.

Our focus on YOLOv6 \cite{li2022yolov6,li2023yolov6}, known for its balance of speed and accuracy, stems from a recognition of the critical need for optimization in edge computing scenarios where resources are scarce yet the demand for high-performance computing is increasingly evident. The drive to refine and enhance the efficiency of such models without compromising their detection capabilities underlines our research. We are particularly inspired by the potential impact of our work on a wide array of applications, from low-power IoT devices to mobile applications requiring real-time analysis and feedback, envisioning a future where advanced object detection is not only possible but also practical and pervasive, regardless of computational limitations.

\subsection{Objectives}

The primary objective of our research is to develop an optimized version of the YOLOv6 \cite{li2022yolov6,li2023yolov6} architecture, termed "Gated YOLO," which incorporates a dynamic gating mechanism to enhance computational efficiency in object detection tasks, particularly in edge computing environments. To achieve this, we aim to:

Implement Dynamic Gating: Integrate a dynamic gating mechanism that adapts to the unique characteristics of input scenes, thereby selectively activating relevant neural pathways and improving model efficiency.

Optimize Through Model Pruning: Apply model pruning techniques in conjunction with dynamic gating to eliminate redundant parameters and streamline the model, focusing computational resources on critical tasks.

Leverage Improved SemHash \cite*{kaiser2018discrete}: Utilize the Improved SemHash technique for effective gate generation during training, allowing for precise control over network activity and further optimization of the model for specific scenarios.

Demonstrate Practical Efficacy: Validate the effectiveness of the Gated YOLO model through extensive testing, focusing on key performance metrics such as processing speed (FPS), computational efficiency (FLOPs), and detection accuracy (mAP@0.5:0.95).

Enhance Real-World Applicability: Ensure that the optimized model maintains high detection accuracy while significantly reducing computational load, making it suitable for deployment in real-world, resource-constrained environments.

Through these objectives, our research seeks to address the pressing challenges of deploying sophisticated object detection models in edge computing scenarios, offering a pathway to more efficient, accurate, and accessible real-time object detection technologies.

\clearpage

